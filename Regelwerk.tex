\documentclass[a4paper, 12pt]{article}
\

\usepackage[ngerman]{babel}
\usepackage[T1]{fontenc}
\usepackage{amsmath}
\usepackage{enumitem}
\usepackage{makecell}

\title{Beerpongturnier Manifest}
\author{Kevin Haberl mit Andreas Grill}
\date{\today}

%Variablen
\newcommand{\startgebuehr}{zehn Euro }
\newcommand{\zeitlimit}{zwanzig Minuten }

\newcounter{arti}
\setcounter{arti}{1}
\newcounter{absatz}
\setcounter{absatz}{1}
\newcounter{upper}
\setcounter{upper}{1}
\setcounter{secnumdepth}{-1}

\newenvironment{artikel}[1]{%
    \subsubsection{Art. \arabic{arti} {#1}.}
    \addtocounter{arti}{1}
    \newcommand{\up}{\textsuperscript{\arabic{upper}}\addtocounter{upper}{1}}
    \newcommand{\abs}{(\arabic{absatz})\addtocounter{absatz}{1}\setcounter{upper}{1}}
}{\setcounter{upper}{1}\setcounter{absatz}{1}}

\newenvironment{para}[1]{%
    \subsubsection{§ \arabic{arti} {#1}.}
    \addtocounter{arti}{1}
    \newcommand{\up}{\textsuperscript{\arabic{upper}}\addtocounter{upper}{1}}
    \newcommand{\abs}{(\arabic{absatz})\addtocounter{absatz}{1}\setcounter{upper}{1}}
}{\setcounter{upper}{1}\setcounter{absatz}{1}}

\begin{document}

\maketitle
\newpage
\tableofcontents

\newpage

\section{Einführung}
Lieber Leser,

\newpage

\section{Bayerisches Beerponggesetz (BayBPG)}
Das bayerische Beerponggesetz in der Fassung vom \today.
\newpage
\subsection{\centering Erstes Buch - Allgemeiner Teil}
\begin{artikel}{Grundsatz}
\abs{} Dieses Buch stellt ein allgemeines Regelwerk für den Beerpongsport dar.
\\\abs{} Die Anwendung von hausinternen Regeln ist beim Beerpong stets erwünscht
\end{artikel}
\newpage
\subsection{\centering Zweites Buch - Der Aufbau}

\begin{artikel}{Spieltisch}
\abs{} \up{}Der Aufbau eines Spieltisches ist auf Abbildung 1.1 beschrieben. \up{}Er darf während des Turniers nicht verändert werden.
\\\abs{} Der Aufbau besteht aus:
\begin{enumerate}
\item einer Mittellinie
\item einer handfreien Zone
\item je 10 Redcups auf jeder Seite der Pyramide
\item beliebig vielen Tischtennisbällen, die zu Beginn des Truniers im Wassercup bereitliegen
\item einem Wassercup zum Waschen der Bälle
\end{enumerate}
\end{artikel}

\begin{artikel}{Spielgerät}
\abs{} Spielgerät und Bierbehältnis ist der Redcup.
\\\abs{} Füllmenge und Inhalt der Redcups kann von jedem Spieler selbst bestimmt werden, sollte aber mindestens in Summe einen halben Liter pro Teammitglied ergeben und pro Becherfüllung nicht die zweite Einkerbung der Redcups unterschreiten.
\\\abs{} Nachfüllen nach Spielbeginn ist verboten
\\\abs{} "muss in BtPO
\end{artikel}

\begin{artikel}{Handfreie Zone}
\abs{} Die handfreie Zone ist eine rechteckige Markierung auf dem Tisch, in welcher die Redcups platziert werden.
\\\abs{} \up{}In und über der handfreien Zone ist es nicht erlaubt die Hände hineinzuhalten und / oder den Ball zu berühren.
\up{}Dabei gelten folgenden Ausnahmen:
\begin{enumerate}
\item Der Ball wurde noch nicht von der gegenerischen Mannschaft unter Kontrolle gebracht.
\item Der geworfenen Ball springt nicht mehr höher als der Rand des Redcups.
\item Der geworfenen Ball rollt oder ist zum Erliegen gekommen.
\item Der Ball hat in der Wurfbewegung die Hand des gegnerischen Spielers nicht verlassen.
\end{enumerate}
\end{artikel}
\begin{artikel}{Mittellinie}
\up{}Die Mittellinie gilt als Abtrennung der beiden Spielfeldhälften. \up{}Sie durchzieht dabei imaginär den kompletten Raum.
\end{artikel}

\begin{artikel}{Ellbogen}
Bei einem Wurf darf der Ellbogen die Tischkante nicht überragen.
\end{artikel}

\begin{artikel}{Spielball}
\abs{} Als Spielball ist ein Tischtennisball mit einem Umfang von 40mm zu verwenden.
\\\abs{} "BtPO"
\end{artikel}

\begin{artikel}{Videobeweis}
\abs{} Sollte die Möglichkeit eines Videobeweises bestehen, ist dieser nur während des Finalspiels zulässig.
\\\abs{} \up{}Kann eine Situation weder von Turnierleitung noch von der Mehrheit der Zuschauer eindeutig geklärt werden, kann pro Team einmalig die fragliche Situation gechallenged werden. \up{}Ein Mitglied der Turnierleitung unterbricht mit einem Handzeichen das Spiel, überprüft, gegebenenfalls mit mehreren Mitgliedern der Turnierleitung die Situation anhand der Videoaufzeichnungen in der Leitstelle und entscheidet abschließend über die Spielfortsetzung.
\end{artikel}

\newpage
\subsection{\centering Drittes Buch - Das Spiel}

\begin{artikel}{Spielbeginn}
\abs{} Jedes Team bekommt bei jedem neuen Spiel einen Tisch zugewiesen. 
\\\abs{} \up{}Für die Seitenwahl sind die Teams selbst verantwortlich. \up{}Bei Differenzen entscheidet der Verlierer der Erstwurf-Entscheidung.
\end{artikel}

\begin{artikel}{Erstwurf-Entscheidung}
\up{}Um zu entscheiden, wer zu Beginn des Spiels den ersten Wurf tätigen darf, tritt vor Beginn des Spiels jeweils ein Spieler der beiden Teams zu einem Duell "Schere-Stein-Papier" an. \up{}Der Sieger erhält den ersten Wurf.
\end{artikel}

\begin{artikel}{Der Wurf}
Der Ball gilt als geworfen, wenn er die Hand des Werfers in der Wurfbewegung nicht mehr berührt.
\end{artikel}

\begin{artikel}{Anzahl der Bälle}
\abs{} Grundsätzlich wird ein Spiel mit zwei Bällen bestritten.
\\\abs{} Es kann jederzeit auf einen Ball ausgewichen werden, um die Chancengleichheit zu gewährleisten.
\end{artikel}

\begin{artikel}{Spielverlauf}
\abs{} \up{}Geworfen werden muss bei Anwesenheit beider Partner immer abwechselnd. \up{}Als Ausnahme gilt der Rebound.
\\\abs{} \up{}Ein Partner darf sich während des Spiels aus triftigen Gründen für maximal drei Würfe der gegnerischen Mannschaft vom Tisch entfernen. \up{}In dieser Zeit spielt sein Partner alleine. \up{}Triftige Gründe sind:
\begin{enumerate}
\item Klogang.
\item Medizinische Grundversorgung
\item Genehmigung der Turnierleitung
\end{enumerate}
\abs{} \up{}Bei einem Treffer muss das gegnerische Team den Inhalt des Redcups trinken und den Becher außerhalb der handfreien Zone am Tand des Biertisches abstellen. \up{}Ein Wurf gilt dann als Treffer, wenn der Tischtennisball in der Flüssigkeit des Redcups liegen bleibt.
\\\abs{} \up{}Berührt der Ball während des Fluges ein anderes Objekt und landet trotzdem oder aus diesem Grund in einem Redcup, gilt dies als Treffer. \up{}Becher, welche durch den Einfluss des Balls oder des Gegners zu Fall kommen, zählen als Treffer.
\\\abs{} \up{}Der Ball darf nach einem Wurf des Gegners erst hinter oder im seitlichen Aus der Tischplatte gefangen werden. \up{}Landet der Ball nach einem gescheiterten Fangversuch des Gegners im Becher zählt dies als Treffer.
\end{artikel}

\begin{artikel}{Rearrangement (Re-Rack)}
\abs{} \up{}Bei genau sechs oder drei verbleibenden Redcups im gegnerischen Team kann das werfenden Team fordern, dass die Recups wieder in einer pyramidalen Form angeordnet werden. \up{}Das Fordern muss vor dem nächsten eigenen Wurf der Mannschaft erfolgen und ist eine Hohlschuld. \up{}Die Aufstellung der Reducups erfolgt anhand der Spitze der Pyramide
\\\abs{} Ein Einfordern bei einer anderen Anzahl an Redcups als der in Absatz 1 genannten, entfaltet keine Wirkung.
\end{artikel}

\begin{artikel}{On Fire}
\abs{} \up{}Hat sich die Becheranzahl einer Mannschaft auf drei oder weniger Becher reduziert und die gegnerische Mannschaft hat 5 oder mehr Becher auf dem Tisch, kann die Mannschaft mit drei Becher nach einem Treffer "On Fire" fordern. \up{}Hierbei ist die Mannschaft mit drei verbleibenden Bechern solange am Zug bis entweder
\begin{enumerate}
\item die werfende Mannschaft ihre Trefferserie untbricht
\item[oder]
\item die gegnerische Mannschaft ebenfalls nur noch drei Becher vorweisen kann.
\end{enumerate}
\abs{} \up{}On Fire ist eine Hohlschuld. \up{} Ein Rebound hebt die On Fire Wirkung nicht auf.
\end{artikel}

\begin{artikel}{Rebound}
\abs{} Prallt der Ball nach einem getätigten Wurf con der gegnerischen Seite zurück und überquert dabei die Mittellinie in vollem Umfang, so darf die werfenden Mannschaft den Wurf erneut tätigen.
\abs{} \up{}Mit erneutem Überqueren der Mittellinie in die ursprüngliche Richtung geht die Wurfberechtigung wieder auf die gegenerische Mannschaft über. \up{}Dies gilt nicht, wenn die Mannschaft, auf welcher Seite sich der Ball zum Zeitpunkt der Entscheidung befindet, den Ball unter Kontrolle gebracht hat.
\end{artikel}

\begin{artikel}{Spielabbruch}
\abs{} "BtPO"
\\\abs{} Die Turnierleitung kann ein Spiel abbrechen, wenn in absehbarer Zeit nicht vermutet werden kann, dass eine sichere Durchführung des Spiels möglich ist.
\\\abs{} \up{}Ein abgebrochenes Spiel wird 0:0 gewertet.
\\\abs{} "BtPO"
\\\abs{} "BtPO"
\end{artikel}

\begin{artikel}{Spezielle Würfe}
\up{}Die folgenden Artikel finden in einem Spiel, bei dem mit einem Ball gespielt wird, keine Anwendung
\end{artikel}

\begin{artikel}{Air-Ball}
\abs{} Als Air-Ball wird der Ball bezeichnet, welcher beim Wurf die hintere Kannte des Tisches überfliegt ohne den Tisch dabei zu berühren.
\\\abs{} Die Anzahl der Air-Balls pro Spiel werden summiert.
\\\abs{} \up{}Bei einer Änderung der Anzahl der Summer der Air-Balls eines Teams, muss das Gegnerteam mit einem klaren Handzeichen die neue Summe der Air-Balls anzeigen. \up{}Wird die Anzeige bis zum nächsten Wurf des Gegnerteams nicht durchgeführt, erhöht sich die Air-Ball Zahl nicht.
\end{artikel}

\begin{artikel}{Bomb}
\up{}Als Bomb wird der Spielzug bezeichnet, bei dem beide Bälle von einem Team in den selben Redcup getroffen werden.
\end{artikel}

\begin{artikel}{Herausblasen/Herausfingern}
\up{}Kreist ein Ball nach einem Wurf an der Wand des Redcups noch einige Male, bevor er die Flüssigkeit oder einen bereits darin befindlichen Ball berührt, darf dieser während dieser Zeitdauer aus dem Redcup herausgeblasen oder herausgefingert werden. \up{}Fliegt der Ball beim Herausblasen in einen anderen, im Spiel befindlichen Redcup, so zählt dies als Treffer. \up{}Wird der Ball herausgeblasen oder herausgefingert, nachdem er die Flüssigkeit berührt hat, zählt dies als Treffer.
\end{artikel}

\begin{artikel}{Balls-Back}
\abs{} \up{}Treffen in einem Spiel mit zwei Bällen beide Teampartner, so kann das Team vor den Würfen des Gegnerteams Balls-Back verlangen. \up{}Hierbei hat das werfende Team erneut zwei Würfe, muss diese aber per Trick-Shot ausführen. \up{}Der Trick-Shot muss vor dem Wurf benannt werden. \up{}Als Trick-Shot gilt jeder Wurf welcher nicht durch eine normale Wurfbewegung oder mit offenen Augen durchgeführt wird.
\\\abs{} Balls-Back ist eine Hohlschuld.
\end{artikel}

\begin{artikel}{Deathcup}
\abs{} \up{}Wird ein bereits getroffener Becher von der gegenrischen Mannschaft aus dem Grund des Getränkekonsums oder des Wegstellens in der Hand gehalten, so kann die werfenden Mannschaft versuchen diesen Becher erneut zu treffen. \up{}Wird der Becher getroffen ist das Spiel zu Ende.
\\\abs{} Das Spiel wird -Becher der werfenden Mannschaft- zu 0 gewertet.
\end{artikel}

\begin{artikel}{Redemption}
\abs{} Wird der letzte BEcher einer Mannschaft getroffen, so bekommt die verlierende Mannschaft einen Nachwurf (Redemption).
\\\abs{} Trifft die Mannschaft alle übrigen Becher der gegnerischen Mannschaft ohne Fehlwurf so erfolgt eine Overtime mit jeweils drei Becher.
\end{artikel}

\begin{artikel}{Bouncer}
\abs{} Ein geworfener Ball, der mindestens einmal die Tischplatte berührt, bevor er einen Becher berührt, ins Seitenaus oder ins hintere Aus fliegt, nennt man Bouncer.
\\\abs{} Der Ball darf, nachdem er in einen Bouncer übergegangen ist, mit der Hand weggeschlagen werden.
\\\abs{} Trifft der Bouncer, zählt der Treffer doppelt und zwei Becher müssen vom Spielfeld entfernt werden.
\end{artikel}

\begin{artikel}{Sieger des Spiels}
\abs{} \up{}Sieger des Spiels ist das Team, welches als erstes alle gegnerischen Becher getroffen hat oder nach Ende der Zeit mehr eigene Becher verzeichnen kann. \up{}Der Verlierer des Spiels ist nicht verpflichtet die restlichen vollen Becher zu trinken, jedoch wird dies aus sportlicher Sicht gerne gesehen.
\\\abs{} Wird bei einem Unentschieden in letzter Sekunde ein Treffer erzielt, so zählt dieser nur, wenn sich der Ball bereits bei Verstreichen der letzten Sekunde im Flug befunden hat.
\end{artikel}

\newpage
\setcounter{arti}{1}
\section{Beerpongturnier Strafgesetzbuch (BStGB}
Das Beerpongturnier Strafgesetzbuch nach der Fassung vom \today durch die Vorstandschaft des Beerpongturniers.
\newpage
\begin{para}{Keine Strafe ohne Tat}
Eine Tat kann nur bestraft werden, wenn ein Fehlverhalten nach diesem Gesetz vorliegt.
\end{para}

\begin{para}{Zeit der Tat}
\abs{} \up{}Eine Tat ist zu der Zeit begangen, zu welcher der Spieler oder Anwesende gehandelt hat oder im Falle des Unterlassens hätte handeln müssen. \up{}Wann und ob Erfolg eintritt, ist nicht maßgebend.
\end{para}

\begin{para}{Ort der Tat}
\abs{} Der Ort, an welchem der Täter die Tat verübt oder verüben will, ist der Tatort.
\\\abs{} Für die Taten, welche in diesem Gesetz geregelt sind, ist der Ort grundsätzlich innerhalb des Turnierort und ein Umkreis von 15 Kilometer.
\end{para}

\begin{para}{Personen und Sachbegriffe}
\abs{} Im Sinne dieses Gesetzes ist
\begin{enumerate}
\item der Spieler:
\\wer die Teilnahmegebühr bezahlt;
\item die Turnierleitung:
\\wer die Leitung und Führung des Turniers nach der Beerpongturnierprozessordnung übernimmt;
\item das Schiedsgericht:
\\wer die Gerichtsbarkeit des Turniers nach der Beerpongturnierprozessordnung vertritt;
\item Teilnehmer:
\\wer am Turniertag 0:00 Uhr bis zum darrauffolgenden Tag 24:00 Uhr am Ort der Tat anwesend ist;
\item das Turnier:
\\der Zeitraum von Beginn der Pre-Show bis zum Ende der Siegerehrung
\item der Täter:
\\wer als Spieler oder Turnierteilnehmer eine Straftat begeht.
\end{enumerate}
\end{para}

\begin{para}{Grundsatz der Strafbemessung}
\abs{} \up{}Die Schuld des Täters ist Grundlage der Strafbemessung. \up{}Die Auswirkung auf seine Turnierzukunft ist zu berücksichtigen.
\\\abs{} \up{}Bei der Zumessung wägt die Turnierleitung oder bei einem Schiedsgerichtverfahren das Schiedsgericht die Umstände, die für und gegen den Täter sprechen gegeneinander ab. \up{}Dabei kommen besonders folgende Merkmale in Betracht:
\begin{enumerate}
\item die Beweggründe und die Ziele des Täters, besonders rassistische, fremdenfeindliche, antisemitische und sonstige menschenverachtende,
\item die Gesinnnung, die aus der Tat spricht,
\item die Härte der Tat,
\item die Art der Ausführung,
\item die Vorstrafen des Täters,
\item das Verhalten nach der Tat, besonders sein Bemühen, den Schaden wiedergutzumachen.
\end{enumerate}
\abs{} \up{} Wer eine Tat begeht, die durch Notwehr geboten ist, handelt nicht rechtswidrig. \up{} Notwehr ist die Verteidigung, die erforderlich ist, um einen gegenwärtigen rechtswidrigen Angriff von sich oder einem anderen abzuwenden.
\\\abs{} Der Versuch ist nur bei persönlichen Strafen strafbar.
\\\abs{} Als Anstifter wird gleich dem Täter bestraft, wer vorsätzlich einen anderen zu dessen vorsätzlich begangener Tat bestimmt hat.
\\\abs{} Als Gehilfe wird bestraft, wer vorsätzlich einem anderen zu dessen vorsätzlich begangener rechtswidrigen Tat Hilfe geleistet hat.
\end{para}

\begin{para}{Persönliche Strafen und Spielstrafe}
\abs{} Mit einer persönlichen Strafe, werden Taten bestraft, die nicht im Zusammenhang mit dem Turnier im engeren Sinne stehen.
\\\abs{} Mit einer Spielstrafe werden Taten bestraft, welche im inneren ZUsammenhang mit dem Turnier stehen (Regelbruch). Spielstrafen können nur an Spieler und nur während des laufenden TUrniers verhängt werden.
\\\abs{} \up{}Der innere Zusammenhang im engeren Sinn besteht bei der Pre-Show, den Spielen und der Siegerehrung. \up{}Der innere Zusammenhang ist nicht mehr gegeben, wenn das Motiv der Tat über das Spiel hinaus in den privaten Geltungsbereich des Betroffenen eintritt.
\end{para}

\begin{para}{Arten der persönlichen Strafe}
Eine persönliche Strafe liegt vor, wenn
\begin{enumerate}
\item ein Teilnehmer einen anderen Teilnehmer mutwillig beleidigt.
\item ein Teilnehmer aufgrund eines zu hohen Alkoholkonsums einschläft, ins Koma fällt, unkontrolliert uriniert, defäkiert oder erbricht.
\item ein Teilnehmer mit augenscheinlichem Alkoholgehalt von mehr als 0,5 Promille am Turniertag sowie dem Tag danach 09:00 Uhr Ortszeit ein Kraftfahrzeug oder ein Fahrrad steuert.
\item ein Teilnehmer einen anderen Teilnehmer absichtlich verletzt
\item eine sonstige, nach der allgemeinen Verkehrssitte verwerfliche Straftat begeht, welche nicht im inneren Zusammenhang mit dem Turnier steht.
\end{enumerate}
\end{para}

\begin{para}{Arten der Spielstrafe}
Eine Spielstrafe liegt vor, wenn
\begin{enumerate}
\item ein Spieler einen beeinflussten Wurf ausführt.
\item ein Spieler einen Regelbruch begeht.
\end{enumerate}
\end{para}

\begin{para}{Beeinflusster Wurf}
Als beeinflusster Wurf gilt ein Wurf, wenn
\begin{enumerate}
\item der ausführende Spieler durch ein Objekt oder Tier im Wurf nachweislich beim Wurf behindert wurde.
\item der ausführende Spieler durch ein unvorhergesehenes oder unabsichtlich herbeigeführtes Ereignis, das auf seinen Körper einwirkt beim Wurf behindert wurde.
\end{enumerate}
\end{para}

\begin{para}{Regelbruch}
\abs{} Ein Regelbruch liegt vor, wenn eine Vorschrift des Bayerischen Beerponggesetz verletzt wird
\\\abs{} Ein unmittelbarer Regelbruch liegt vor, wenn
\begin{enumerate}
\item ein Spieler seine Hand in die handfreie Zone hält, während der Gegner wirft oder vom Ball in der handfreien Zone getroffen wird, auch wenn der Ball offensichtlich nicht im Redcup gelandet wäre. Der Zeitpunkt des Balls beim Verlassen der Wurfhand ist entscheidend;
\item eine Mannschaft sich weigert die Becher zu einer Pyramide anzuordnen;
\item ein Spieler einen oder mehrere Becher während des laufenden Spiels nachfüllt;
\item ein Spieler mutwillig einen gegnerischen Becher umwirft;
\item ein Spieler mutwillig das Spiel verzögert;
\item ein Spieler wiederholt Mitglieder der Presse oder Turnierleitung beschimpft oder beleidigt;
\item ein Spieler die Leitstelle betritt;
\item ein Spieler in einem Spiel mit Zeituhr nach Ablauf von 40 Sekunden nach nicht geworfen hat.
\end{enumerate}
\abs{} Ein mittelbarer Regelbruch liegt vor, wenn ein Spieler einen sonstige allgemeine Spielregel bricht.
\end{para}

\begin{para}{Strafmaßnahmen}
\abs{} Ein Extra- oder Wiederholungswurf wird vergeben, wenn ein beeinflusster Wurf durchgeführt wurde.
\\\abs{} Ein Strafbecher wird vergeben, wenn ein Regelbruch begangen wurde.
\\\abs{} Eine Matchstrafe liegt vor, wenn der dritte Strafbecher gegen eine Mannschaft vergeben wird.
\\\abs{} Ein Turnierausschluss liegt vor, wenn eine persönliche Strafe vorliegt oder ein Spieler
\begin{enumerate}
\item einen unmittelbaren Regelbruch
\item[oder]
\item wiederholt einen Spielbogenvermerk erhält
\end{enumerate}
\end{para}

\begin{para}{Extra- oder Wiederholungswurf}
Wer einen Extra- oder Wiederholungswurf erhält, ist berechtigt einen erneuten Wurfversuch zu tätigen.
\end{para}

\begin{para}{Strafbecher}
\up{} Ein Strafbecher ist durch die Turnierleitung aus den noch vorhandenen Redcups des Teams des Täters zu wählen. Ist der letzte Becher ein Strafbecher, ist das Match beendet und nach Becherstand zu werten.
\end{para}

\begin{para}{Matchstrafe}
Wer eine Matchstrafe erhält, verliert das Spiel mit der Wertung X zu 0 Becher, wobei X für die verbleibenden Becher des Gegners steht
\end{para}

\begin{para}{Turnierausschluss}
\abs{} Wer einen Turnierausschluss erhält, darf nicht mher am Turnier teilnehmen.
\\\abs{} Der ausgeschlossenen Teilnehmer hat mit sofortiger Wirkung den Turnierort zu verlassen.
\end{para}

\begin{para}{Fairplay}
\up{}Stimmen zwei Drittel der anwesenden Spieler gegen einen Turnierausschluss, so wird lediglich ein Spielbogenvermerk vorgenommen.
\up{}Ein Ausschluss nach wiederholtem Spielbogenvermerk bleibt unberührt.
\end{para}

\begin{para}{Sperre}
\abs{} Wer eine Sperre erhält, ist für bis zu drei Turniere nicht teilnahmeberechtigt.
\\\abs{} \up{}Erhält ein Teilnehmer nach Wiedereintritt in das Turnier in den folgenden zehn Turnieren erneut eine Sperre, wird der Spieler dauerhaft von der Teilnahme ausgeschlossen (Turnierausschluss auf Dauer). \up{}Diese Sperre kann nur durch ein Schiedsgerichtsurteil ausgesprochen oder aufgehoben werden.
\end{para}

\newpage
\section{Beerpongturnierprozessordnung (BtPO)}
\setcounter{arti}{1}
Beerpongturnierprozessordnung aus der allgemeinen Bekanntmachung vom \today durch die Vorstandschaft des Beerpongturniers.
\newpage
\subsection{\centering Erstes Buch - Allgemeiner Teil}

\begin{para}{Grundsatz}
\abs{} \up{}Dieses Gesetz regelt ein Beerpongturnier. \up{}Alle nachfolgenden Paragraphen gelten für ein Turnier, dass nach den Grundsätzen des Beerpongturniers gespielt wird, welches sich aus dem harten Kern entwickelt hat.
\\\abs{} \up{}Jeder Paragraph dieses Gesetzes hat uneingeschränkte Wirkungskraft. \up{}Wird die Wirkung eines Paragraphen durch einen anderen Paragraphen eingeschränkt, so hat immer der erstaufgeführte Paragraph Vorrang. \up{}Fristen werden gemäß §§187 ff. BGB angewendet.
\\\abs{}Die Spiele werden nach Maßgabe des Bayerischen Beerponggesetzes durchgeführt.
\end{para}

\begin{para}{Aufbau und Organisation}
\abs{} Die Turnierorganisation umfasst die Vorstandschaft, den Gastgeber, die Pressemitglieder, die Turnierleitung sowie ein Schiedsgericht.
\\\abs{} \up{}Die Vorstandschaft steht der kompletten Turnierorganisation vor. \up{}Sie bestellt die Turnierleitung. \up{}In schwerwiegenden Entscheidungen erweitert die Vorstandschaft die Turnierleitung.
\\\abs{} Die Turnierorganisation ist in drei Teile untergliedert.
\\\abs{} \up{}Die Vorstandschaft stellt die Spielregeln auf und stellt im Vorfeld mit dem Gastgeber die Rahmenbedingungen des Turniers zur Verfügung. \up{}Die Turnierleitung ist angehalten für einen reibungslosen Ablauf des Turniers zu sorgen. \up{}Das Schiedsgericht bemüht sich um die gerechte Beurteilung von Regelvergehen.
\end{para}

\begin{para}{Gastgeber}
\abs{} \up{}Der Gastgeber ist Veranstalter des Turniers und stellt den Turnierort bereit. \up{}Ist der Turnierort von der Turnierleitung festgelegt, übernimmt diese das Amt des Veranstalters oder delegiert das Amt auf eine Person weiter.
\\\abs{} Den organisatorischen Anweisungen des Gastgebers Folge zu leisten.
\end{para}

\begin{para}{Vorstandschaft}
\abs{} Die Vorstandschaft ist übergeordnetes Organ der Turniere.
\\\abs{} Die Vorstandschaft bestimmt die Turnierleitung, hält Vorstandssitzungen zur Planung der Turniere ab und organisiert die Rahmenbedingungen.
\\\abs{} Ein Sitz in der Vorstandschaft schließt einen Sitz in der Turnierleitung nicht aus.
\\\abs{} \up{}Der Vorstand wird einmalig gewählt. \up{}Um den amtierenden Vorstand abzusetzen und eine Neuwahl einzuberufen, sind am nächsten Turnier die Stimmen von zwei Drittel aller anwesenden Spieler nötig, die in Summe mindestens 70 v. H. der aktiven Spieler darstellen.
\end{para}

\begin{para}{Turnierleitung}
\abs{} \up{}Die Turnierleitung ist leitendes Organ des Turniers.\up{}Sie trägt Sorge für eine ordentliche Organisation und einen geregelten Ablauf des Turniers.
\\\abs{} Allein der Turnierleitung obliegt es weiterreichende Maßnahmen anzuwenden.
\\\abs{} Sollte die Turnierleitung kurzfristig einen vorübergehende Regeländerung bekannt geben, welche dem geregelten Ablauf des Turniers dient, betrifft diese auch laufende Spiele mit sofortiger Wirkung.
\\\abs{} Strafen dürfen nur durch ein Mitglied der Turnierleitung verhängt werden.
\\\abs{} \up{}Der zeitlich beschränkte oder generelle Turnierausschluss darf nur durch ein einstimmiges, positives Votum der kompletten Turnierleitung erfolgen. \up{}Der Ausschluss muss dem auszuschließenden Spieler schriftlich übermittelt werden. \up{}Erfolgt ein mündliches Turnierausschluss während eines laufenden Turniers, so muss die schriftliche Bestätigung spätestens zwei Wochen nach dem Turniertag dem ausgeschlossenen Spieler zugestellt werden.
\\\abs{} \up{}Die Turnierleitung ist angehalten für alle Fragen und Maßnahmen während des Turniers uneingeschränkt zur Verfügung zu stehen. \up{}Sind Mitglieder der Turnierleitung selbst als Spieler an Spielen beteiligt, so sind sie für Fragen aller Art nicht zuständig.
\end{para}

\begin{para}{Pressemitglieder}
\up{}Falls gewünscht, kann die Vorstandschaft zum Zwecke der Archivierung des Turniers in Bild und Ton Pressebeauftragte bestimmen. denen gesonderte Rechte zufallen. \up{}Die gesonderten Rechte bestimmt die Vorstandschaft.
\end{para}

\begin{para}{Vorstandssitzung}
\abs{} Die Vorstandschaft ist angehalten, vor jedem Turnier eine Vorbesprechung mit dem Gastgeber und den Pressemitgliedern abzuhalten.
\\\abs{} In der Vorstandssitzung soll die Umsetzung des Turniers geplant werden und die einzelnen Aufgaben an die verschiedenen Organe des Turniers verteilt werden.
\end{para}

\begin{para}{Leitstelle}
\abs{} \up{}Die Leitstelle ist das organisatorische Zentrum des Turniers. \up{}Sie gewährleistet einen reibungslosen organisatorischen Ablauf und dient als Rückzugsort für die Turnierleitung bei der Bearbeitung des Turnierplans, Steuerung der Elektronik und Ort der Beratungsmöglichkeit in allen Turnierfragen.
\\\abs{} \up{}Der Zutritt zur Leitstelle ist generell nur der Turnierorganisation gestattet.\up{}Es bedarf der Genehmigung eines Mitglieds der Turnierleitung oder der Vorstandschaft, um Zutritt zur Leitstelle zu erhalten.
\end{para}

\begin{para}{Spielplan}
\abs{} \up{}Der Spielplan wird durch die Turnierleitung erstellt. \up{}Er wird bis spätestens eine Stunde vor dem Turnier durch die Turnierleitung festgelegt. \up{}Ja nach Teilnehmerzahl kann dieser in Spielmodus, Gruppenverteilung und Spiellegung variieren. \up{}Er wird in der Leitstelle verwahrt und darf nur von Mitgliedern der Turnierleitung aktualisiert werden.
\\\abs{} \up{}Der Spielplan ist zur Einsichtnahme allen Turnierteilnehmern vorzulegen. \up{}Über die Art der Bereitstellung entscheidet die Turnierleitung.
\end{para}

\begin{para}{Regulärer Eintritt}
\abs{} \up{}Eine Person tritt mit erstmaliger Bezahlung der Startgebühr in den Turnierkreis als Spieler ein. \up{}Grundsätzlich gibt es keine Neuaufnahmen.
\\\abs{} \up{}Unberührt vom Aufnahmestopp sind folgende Eintrittsmöglichkeiten:
\begin{enumerate}
\item Auswahl durch den Gastgeber
\item Entscheidung der Vorstandschaft aufgrund eines Mitgliedsantrags und vorheriger Genehmigung durch den Gastgeber
\item Entscheidung der Vorstandschaft und des Gastgebers.
\item Lebenspartner eines Spielers nach Genehmigung des Gastgebers.
\end{enumerate}
\abs{} \up{}Bei Personen, welche als Lebenspartner eines Spielers oder durch Mitgliedsantrag teilnehmen, lautet die Bezeichung "Spieler auf Zeit. \up{}Alle anderen Teilnehmer tragen den Titel "Spieler auf Lebenszeit".
\end{para}

\begin{para}{Spieler auf Zeit}
Ein Spieler auf Zeit ist solange teilnahmeberechtigt, bis der Eintrittsgrund nicht mehr gegeben ist.
\end{para}

\begin{para}{Spieler auf Lebenszeit}
\up{}Ein Spieler auf Lebenszeit hat lebenslanges Teilnahmerecht. \up{}Dieses Recht kann nur durch einen dauerhaften Turnierausschluss entzogen werden.
\end{para}

\begin{para}{Champion}
\abs{}Als amtierende Champions werden die Spieler bezeichnet, welche als Team das letzte Turnier gewonnen haben. \up{}Die Amtszeit ist der Zeitraum zwischen der Siegerehrung und dem Start des nächsten Turniers.
\\\abs{} Als Champion werden die Spieler bezeichnet, welche das Turnier mindestens einmal gewonnen haben.
\end{para}

\begin{para}{Der goldenen Ball}
\abs{} Der goldene Ball ist die Auszeichnung für den besten Spieler des Turniers.
\\\abs{} \up{}Der beste Spieler des Turniers wird durch die seine Differenz zwischen der Elo-Zahl bei Turnierstart und der Elo-Zahl nach dem Finalspiel ermittelt. \up{}Die Spieler werden nach höchster positiver Differenz absteigend platziert.
\end{para}

\begin{para}{Austritt}
\abs{} Jeder Spieler kann auf eigenen Wunsch den Turnierkreis zu jedem Zeitpunkt verlassen.
\\\abs{} Wird eine Partnerschaft beendet, in welcher ein Spieler auf Zeit involviert ist, so verliert der Spieler auf Zeit seine Teilnahmeberechtigung.
\\\abs{} Einem Spieler auf Lebenszeit kann nur mit einem zeitweiligen oder dauerhaften Turnierausschluss die Teilnahmeberechtigung aberkannt werden.
\\\abs{} \up{}Über die Möglichkeit zur Zulassung zum Wiedereintritt entscheidet in allen Fällen das Schiedsgericht. \up{}Über die Zulassung zum Wiedereintritt entscheidet der Gastgeber.
\end{para}

\begin{para}{Startgebühr}
\abs{} \up{}Der Turnierleitung obliegt es Gebühren für ein Turnier zu verlagen. \up{}Diese Startgebühr muss mindestens einen Monat vor Beginn des Turnier bekanntgegeben werden und darf eine Höhe von \startgebuehr nicht überschreiten. \up{}Der Verwendungszweck muss grundsätzlich nicht veröffentlicht werden.
\\\abs{} \up{}Entscheidet die Vorstandschaft über eine Startgebühr, welche \startgebuehr übersteigt, muss die Höhe der Startgebühr mindestens zwei Monate vor Turnierbeginn veröffentlicht werden. \up{}Hierbei muss der Verwendungszweck der Differenz aus Maximalstartgebühr und tatsächlicher Startgebühr mit Bekanntgabe der Startgebühr veröffentlicht werden.
\\\abs{} Für eine dauerhafte Erhöhung der Startgebühr sind die Stimmen von zwei Drittel aller Spieler nötig, die in Summe mindestens 70 v. H. der aktiven Spieler darstellen.
\\\abs{} \up{}Erlaubte Zahlungsmittel sind PayPal, Banküberweisung oder Barzahlung. \up{}Barzahlung ist zwingend passend am Turniertag zu entrichten.
\end{para}

\begin{para}{Spielvorschrift}
\abs{} \up{}Aufgrund einiger variabler Regeln, kann die Vorstandschaft im Vorfeld des Turniers eine Spielvorschrift erlassen, nach welcher die Spiele angepasst durchgeführt werden.
\end{para}

\newpage
\subsection{\centering Zweites Buch - Der Turnierbetrieb}

\begin{para}{Anmeldung}
\abs{} \up{}Für eine generelle Anmeldung zur Teilnahme am Turnier kann eine Deadline gesetzt werden. \up{}Anmeldungen nach dieser Deadline werden nicht mehr berücksichtigt. \up{}Über Ausnahmen entscheidet der Gastgeber unter Rücksprache mit der Vorstandschaft.
\\\abs{} Vor Beginn des Turnier tragen sich die Teilnehmer zu einem, durch die Turnierleitung gewählten Zeitpunkt bei einer Mannschaftsanmeldung in den Turnierplan ein. 
\\\abs{} \up{}Gespielt wird grundsätzlich in Teams zu je zwei Personen. \up{}Ausnahmen bedürfen der Erlaubnis der Turnierleitung.
\\\abs{} \up{}Ein vorzeitiger Ausstieg aus dem Turnier ist grundsätzlich nicht möglich. \up{}Über Ausnahmen entscheidet die Turnierleitung.
\\\abs{} Nicht in einer Mannschaft gemeldete Spieler sind nicht berechtigt Würfe für diese Mannschaft zu tätigen.
\end{para}

\begin{para}{Auslosung}
\abs{} \up{}Jedem Team wird bei der Auslosung ein Turnierstartplatz zugelost.
\\\abs{} \up{}Die Auslosung erfolgt geheim. \up{}Das Losverfahren legt die Turnierleitung fest.
\\\abs{} Der Losbetrieb wird durch die Turnierleitung koordiniert und überwacht.
\end{para}

\begin{para}{Wahl des Teamnamen}
\abs{} Der Teamname wird von jedem Team selbst festgelegt.
\\\abs{} \up{}Der Teamname ist auf die Teammitglieder festgeschrieben und darf nur von ihnen verwendet werden. \up{}Mehrere Variationen an Teamnamen für eine Besetzung sind jedoch möglich.
\end{para}

\begin{para}{Turniermodus}
\abs{} Der Standardmodus ist -Jeder gegen Jeden-
\\\abs{} Sind mehr als fünf Mannschaften gemeldet, kann auf einen Turniermodus mit Gruppe und/oder K.O.-Phase gewechselt werden.
\end{para}

\begin{para}{Jeder gegen Jeden}
\abs{} IM Modus -Jeder gegen Jeden- treffen alle Mannschaften aufeinander.
\\\abs{} \up{}Sieger ist das Team, welches nach dem letzten Spiel des Turniers die meisten Punkte erzielt hat. \up{}Gibt es einen Gleichstand werden Trefferverhältnis, getroffene Becher und der direkte Vergleich in dieser Reihenfolge verglichen.
\end{para}

\begin{para}{Gruppen- und/oder K.O.-Phase}
\abs{} Jeder Gruppe spielt pro Spieltag ein Spiel.
\\\abs{} \up{}Nach der Gruppenphase werden alle Gruppen zu einer einzigen großen Gruppe zusammengerechnet. \up{}Befinden sich in den Gruppen eine unterschiedliche Anzahl an Teams, werden die Punkte der zahlenmäßig schwächeren Gruppe aufgerechnet.
\\\abs{} Sind mehr als acht Mannschaften gemeldet, so werden für die Vergabe der letzten K.O.-Phasen Plätze Relegationsspiele durchgeführt.
\\\abs{} \up{}Die K.O.-Phase besteht aus Viertelfinale, Halbfinale, Spiel um Platz 3 und dem Finale. \up{}Die Zulosung erfolgt anhand der Platzierung der Gruppenphase. \up{}Dem besten Spieler der Gruppenphase wird somit der schlechteste Spieler der Gruppenphase analog für jedes Spiel der K.O.-Phase zugeteilt.
\\\abs{} Sieger ist das Team, welches das Finalspiel gewinnt.
\end{para}

\begin{para}{Zeitlimit}
\abs{} \up{}Die Vorstandschaft kann für das Turnier ein Zeitlimit pro Spiel festsetzen. \up{}Das Standardzeitlimit beträgt \zeitlimit. \up{}Im Finalspiel gibt es kein Zeitlimit.
\end{para}

\begin{para}{Relegation}
\abs{}Die Anzahl an Relegationsspielen wird durch Anzahl an Teams festgelegt, welche die Grenze von sechs bei einer geraden Anzahl an Teams und fünf bei einer ungeraden Anzahl an Teams überschreiten.
\abs{} Die Gewinner der Relegationsspiele ziehen in das Achtelfinale ein.
\end{para}

\begin{para}{Beginn des Turniers}
\up{}Das Turnier beginnt offiziell mit Beginn des ersten Countdowns der Zeituhr am festgelegten Turniertag. \up{}Es ist gleichzeitig der erste Spielbeginn.
\end{para}

\begin{para}{Pause zwischen Spielen}
\up{}Grundsätzlich gibt es keine Pause zwischen den Spielen. \up{}Die einzelnen Teams tragen Sorge dafür, sobald wie möglich nach dem vorhergegangenen Spiel ihren Spielplatz vorzubereiten. \up{}Vorsätzliche Turnierverzögerung gilt als Spielverzögerung.
\end{para}

\begin{para}{Siegerehrung}
\up{}Ist das Finalspiel des Turniers gespielt und steht ein Sieger fest, so wird die Siegerehrung eingeleitet.
\up{}Hierbei wird entweder von der Vorstandschaft oder den vorherigen Siegern ein Pokal übergeben.
\end{para}

\begin{para}{Gefahrenspiele}
\abs{} Besteht Vorkenntnis das ein Spiel aufgrund der Mentalität der Spieler eskalieren könnte, so wird ein Mitglied der Vorstandschaft als Schiedsrichter bestellt, welcher in Gefahrensituationen eingreift, die den friedlichen Spielverlauf gefährden.
\\\abs{} Muss der Schiedsrichter in ein Spiel eingreifen, so wirft er zum Zeichen der Spielunterbrechung gut sichtbar für beide Mannschaften eine gelbe Flagge auf den Spieltisch und begründet seine Entscheidung.
\end{para}

\begin{para}{Aufzeichnung}
\abs{} \up{}Der Teilnehmer bekennt sich mit der Bezahlung der Startgebühr dazu bereit, die Rechte an Bild und Ton während des Turniers an die Turnierleitung abzutreten.
\\\abs{} \up{}Die Turnierleitung verpflichtet sich Bild- und Tonmaterial auf einer, nur für Turnierteilnehmer zugänglichen Plattform bereitzustellen.
\\\abs{} \up{}Will ein Turnierteilnehmer Bild- oder Tonmaterial öffentlich verwenden, auf dem ein anderer Turnierteilnehmer sichtbare Betrunkenheitsmerkmale aufweist oder der abgebildete Teilnehmer vor dem Turnier ein ausdrückliches schriftliches oder schriftliches Veröffentlichungsverbot bei der Turnierleitung abgegeben hat, bedarf die Veröffentlichung einer expliziten Erlaubnis des abgebildeten Teilnehmers. \up{}Bei Bild- und Tonmaterial, auf welchem mehrere Teilnehmer klar erkennbar sind, muss vor öffentlicher Verwendung jeder erkennbare Spieler mündlich oder schriftlich um Erlaubnis gefragt werden.
\\\abs{} Die Turnierleitung hat dafür Sorge zu tragen, dass eine Auflistung der Veröffentlichungsverbote mit dem Tag, an dem auch die Bild- und Tonaufnahmen bereitgestellt werden, veröffentlicht wird.
\end{para}

\begin{para}{Statistik}
\up{}Die Vorstandschaft ist angehalten eine Statistik über Spieler, Spiele und  Turnierverlauf zu führen. \up{}Diese soll von allen Spielern eingesehen werden können.
\end{para}

\newpage
\subsection{\centering Dritte Buch - Strafbarkeit}

\begin{para}{Verhängung von Strafen}
Strafen werden gemäß dem Beerpongturnier Strafgesetzbuch (BStGB) durch die Turnierleitung verhängt.
\end{para}

\begin{para}{Spielbogenvermerk}
\up{}Wird ein Regelbruch festgestellt und eine Strafe ausgesprochen, wird der Name des Spielers, der den Regelbruch begangen hat, der dazugehörige Teamname und die Art des Regelbruchs vermerkt. \up{}Der Spieler wird von der Turnierleitung verwarnt. \up{}Bei wiederholtem Regelburch kann die Turnierleitung eine Disqualifikation des Teams vornehmen.
\end{para}

\begin{para}{Weiterreichende Maßnahmen}
\abs{} \up{}Gibt es einen Zwischenfall, welcher nicht durch Normen erfasst wird oder wird eine schwerwiegende mittelbare Strafe ausgesprochen, treten mit sofortiger Wirkung weiterreichende Maßnahmen in Kraft. \up{}Diese Maßnahmen geben der Turnierleitung schnelle und weitreichende Entscheidungsgewalt, um einen flüssigen und fairen Turnierverlauf zu gewährleisten.
\\\abs{} \up{}Sobald weiterreichende Maßnahmen eintreten, wird das Turnier durch ein akustisches Signal unterbrochen und erst nach Klärung des Sachverhalts und eventueller Aussprache von Strafen durch die Turnierleitung wieder freigegeben. \up{}Die verstrichenen Zeit wird nachgespielt.
\\\abs{} Während der Turnierunterbrechung
\begin{enumerate}
\item ist die Turnierleitung befugt
\begin{enumerate}
\item Strafen ohne vorhergehenden Regelverstoß auszusprechen.
\item einzelne Spieler ohne Angabe von Gründen zu disqualifizieren (Turnierausschluss).
\item jeden Treffer, jeden Wurf und jede Strafe rückgängig zu machen.
\item in Härtefällen einen Wurf-, Spiel- oder Turnierabbruch durchzuführen.
\end{enumerate}
\item Das Fairplay findet keine Anwendung.
\end{enumerate}
\abs{} Nach jedem Turnier, an dem weiterreichende Maßnahmen notwendig waren, muss das Schiedsgericht über die Verhältnismäßigkeit und sämtliche ausgesprochenen Strafen urteilen.
\end{para}

\begin{para}{Verhältnismäßigkeitsgrundsatz}
Wird eine Strafe von der Turnierleitung ausgesprochen, die offenkundig unverhältnismäßig ist, ist dies nichtig.
\end{para}

\newpage
\subsection{\centering Viertes Buch - Das Schiedsgericht}

\begin{para}{Grundsatz des Schiedsgerichts}
\abs{} Das Schiedsgericht urteilt über Strafen, Sperren, besondere Vorfälle und Turnierabbrüche.
\end{para}

\begin{para}{Schiedsgerichtsverhandlung}
\abs{} \up{}Eine Schiedsgerichtsverhandlung muss nach Antrag innerhalb von zwei Wochen nach Turnierende abgehalten werden. \up{}Der Vorstandsvorsitzende und sein Vertreter haben den Vorsitz, wobei der Vertreter als Vertreter der Interessen des Turniers auftritt. \up{}Bei Turnierausschlüssen ist der Gastgeber anzuhören.
\\\abs{} \up{}Ein Täter sowie Zeugen müssen auf geeignetem Wege angehört werden. \up{}Versäumt dieser die Anhörung wird der Tatbestand als gegeben festgestellt.
\\\abs{} Ein Protokoll der Verhandlung muss auf einem geeigneten Weg veröffentlicht und für alle Mitglieder zugänglich gemacht werden
\end{para}

\begin{para}{Anfechtung}
\up{}Entscheidungen des Schiedsgerichts können bis maximal zwei Wochen nach Bekanntgabe angefochten werden. \up{}Hierbei muss der Anfechtende seinen Stand der Dinge plausibel mit Beweisen darlegen. \up{}Wird eine Entscheidung angefochten und der Anfechtung stattgegeben, muss eine Spielerversammlung mit 70 v. H. der aktiven Spieler vor Beginn des nächsten Turniers einberufen werden, welche über den Ausgang des Falles abstimmt.
\end{para}

\begin{para}{Maßnahmen bei Verhandlungen gegen ein Mitglied der Vorstandschaft}
\abs{} \up{}Wird eine Verhandlung gegen ein Mitglied der Vorstandschaft geführt, scheidet dieses bis zum Urteilsspruch aus dem Schiedsgericht aus. \up{}Seine Position wird vorübergehend durch die nachfolgenden Mitglieder besetzt. \up{}Die Reihenfolge der nachzurückenden Mitglieder ergibt sich aus der Anzahl der Turnierteilnehmer, Elo-Zahl und dem Alter.
\\\abs{} Eine Anfechtung ist nicht möglich.
\end{para}

\newpage
\subsection{\centering Fünftes Buch - Wahrung der Gemeinschaft}

\begin{para}{Grundsatz von der Wahrung der Gemeinschaft}
\abs{} \up{}Die Wahrung der Gemeinschaft ist das wichtigste zu erreichende Gut des Spielers. \up{}Es muss unter allen Umständen verteidigt werden.
\\\abs{} \up{}Ein Spieler hat sich Zeit seines Lebens zu absoluter Verschwiegenheit über Ereignisse der Gemeinschaft verschworen. \up{}Diese Schweigepflicht gilt auch bei einem dauerhaften Turnierausschluss.
\end{para}

\begin{para}{entfällt}
\end{para}

\begin{para}{entfällt}
\end{para}

\begin{para}{entfällt}
\end{para}

\begin{para}{entfällt}
\end{para}

\begin{para}{entfällt}
\end{para}

\begin{para}{Tribute}
\up{}Wird eine Statistil geführt und trennt sich eine Mannschaft, welche bereits mindestens zweimal das Turnier gewonnen hat, so hat für diese Mannschaft eine besondere Ehrung, genannt Tribute, zu erfolgen. \up{}Dies kann in einer kurzen schriftlichen, mündlichen oder filmerischen Danksagung geschehen.
\end{para}

\begin{para}{Hall of Fame}
\abs{} In die Hall of Fame werden Spieler erhoben, welche sich durch besondere Verdienste im oder um das Turnier ausgezeichnet haben.
\\\abs{} Die Erhebung erfolgt alle zehn Turniere.
\\\abs{} Die Erhebung erfolgt durch die Vorstandsmitglieder.
\\\abs{} \up{}Rechtswirksamkeit erlangt die Erhebung mit Aushändigung einer Urkunde.
\end{para}

\begin{para}{entfällt}
\end{para}

\begin{para}{sedes vakanz}
\abs{} Kann ein amtierender Champion nicht am nächsten Turnier teilnehmen, so spricht man vom sedes vakanz.
\\\abs{} \up{}Bei einer sedes vakanz wird der Name des Teams gesperrt und der verbleibende Spieler muss sich einen neuen Teampartner und neuen Teamnamen suchen. \up{}Sind beide Champions nicht anwesend, so wird der Teamname gesperrt und beide dürfen am nächsten Turnier, an welchem beide anwesend sind, nicht miteinander antreten.
\end{para}

\newpage
\section{BPT Elo Wertung}
Die BPT Elo Wertung liegt der Elo-Zahl aus dem Schach zugrunde. Im Nachgang wird die BPT Elo Wertung immer als Elo-Zahl bezeichnet.
\subsection{Legende}
\begin{itemize}
\item \(R\) = Elo-Zahl des Spielers
\item \(R'\) = Neue Elo-Zahl
\item \(A\) = Alte BPT Spielstärke
\item \(Q\) = Umrechnungskonstante
\item \(E\) = Erwartungswert
\item \(k\) = k-Faktor
\item \(S\) = Punkte aus Punkteverteilung
\end{itemize}
\subsection{Elo Startwert}
Seit dem BPT XIX wird der Elo-Wert nicht mehr anhand der alten Spielstärke berechnet. Als Startwert wird nun eine feste Elo-Zahl von 1800 vergeben. Ab dieser Zahl wird die Turnierentwicklung simuliert und nachgestellt.
Um einen belegbaren und aussagekräftigen Einstieg in das Turnier zu gewährleisten, werden aktive Spieler bei der ersten Teilnahme immer auf den letzten Platz der Rangliste gesetzt.
\subsection{Berechnung des Erwartungswerts}
Der Erwartungswert \(E\) gibt die Sieg-Wahrscheinlichkeit des jeweiligen Spielers an und ist nötig um die neue Elo-Zahl nach jedem Spiel zu berechnen.
\begin{align}
E&=\{x|0 \leq x \leq 1\}
\end{align}
\(Y\) stellt den Abstufungsumfang der Elo-Skala von stark zu schwach dar. Diese kann - wenn notwendig - angepasst werden. Vorerst wird für das BPT Rating die Abstufung
\begin{align}
Y&=700
\end{align}
festgelegt. Sie ergibt sich aus der Elo-Spanne zwischen stärkstem und schwächstem Spieler.

\subsubsection{Erwartungswert des Spieler A}
\begin{align}
E_{A} = \frac{1}{1+10^{(R_{B}-R_{A})/Y}}
\end{align}

\subsubsection{Erwartungswert des Spieler B}
\begin{align}
E_{B} = \frac{1}{1+10^{(R_{A}-R_{B})/Y}}
\end{align}

Somit ergibt sich aus den beiden Erwartungswerten
\begin{align}
E_A + E_B = 1
\end{align}

\subsection{Neuberechnung der Elo-Zahl}
\subsubsection{\(k\)-Faktor}
Der \(k\)-Faktor gibt an, wie viele Elo-Punkte ein Spieler bei einer Partie maximal hinzugewinnen kann.
Wird der \(k\)-Faktor sehr hoch angesetzt, wirken sich die zufällige Einzelergebnisse, welche nicht der normalen Erwartung entsprechen sehr stark aus. Damit schwankt die Elo-Zahl sehr stark.
Ist der \(k\)-Faktor sehr niedrig angesetzt, erfolgt eine Elo-Anpassung sehr träge. Viele Spiele sind also nötig um eine signifikante Änderung der Elo-Zahl zu erreichen.
\subsubsection{Schachmodell}
\begin{align}
k=40
\\k=20
\\k=10
\end{align}
Im dreistufigen Schachmodel werden für neue Spieler mit weniger als 30 Partien (6) angesetzt.
Für jeden regulären Spieler mit mindestens 30 Partien und \(R <= 2400\) wird (7) angesetzt. Dies trifft auf die allermeisten Spieler zu.
Im Top-Bereich von \(R > 2400 \) ist (8) anzusetzen.
\subsubsection{Schweizer Tischtennis Model}
Im Tischtennismodel des Schweizer Verbands wird \(k=10\) für alle Spieler angesetzt.
\subsubsection{Modell des Beerpongturniers}
\begin{align}
k=90
\\k=70
\\k=30
\end{align}
Aufgrund der wenigen Spiele pro Jahr wurden in einem dreistufiges Model die \(k\)-Faktoren für neue Spieler mit weniger als 30 Partien (9) festgesetzt. Für die breite Masse aller Spieler mit mindestens 30 Partien und \(R< 2000\) wird (10) verwendet. Für die Top-Spieler mit \(R>=2000>\) wird (11) angesetzt.
\subsubsection{Punkteverteilung}
Der Wert \(S\) gibt die Punkte an, welcher ein Spieler für den Spielausgang erhält
\begin{itemize}
\item \(S = 1\) für einen Sieg
\item \(S = 0.5\) für ein Unentschieden
\item \(S = 0\) für eine Niederlage
\end{itemize}

\subsubsection{Neuberechnung}
Die Neuberechnung der Elo-Zahl ergibt sich aus:
\begin{align}
R' = R + k\times(S-E)
\end{align}

\subsection{Berechnungsmodell für das Beerpongturnier im Teamwettbewerb}
Das Beerpongturnier ist als Teamwettbewerb aufgebaut. Dies macht die Umsetzung eines Einzelwertungssystems aufwendiger. Jedes Team besteht aus 2 oder 3 Spielern.
Es wird angenommen, dass jeder Spieler für sich gegen die jeweiligen anderen Spieler der gegnerischen Mannschaft spielt.
\subsubsection{Konstellationen der Elo-Berechnung bei einem Teamwettbewerb}
\begin{tabular}{lr}
H1 = Heimspieler 1 & G1 = Gastspieler 1 \\
H2 = Heimspieler 2 & G2 = Gastspieler 2 \\
H3 = Heimspieler 3 & G3 = Gastspieler 3 \\
\end{tabular}
\subsubsection{2 vs. 2 (Standardfall)}
\begin{tabular}[h]{ll}
H1 vs. G1 & 2 Berechnungen \\
H1 vs. G2 & 2 Berechnungen \\
H2 vs. G1 & 2 Berechnungen \\
H2 vs. G2 & 2 Berechnungen \\
\end{tabular}
\\
\\Für jede Begegnung sind sowohl die Berechnung für den Heimspieler als auch die Berechnung für den Gastspieler mit den beiden Erwartungswerten nötig. Somit sind insgesamt acht Berechnungen nötig.

\subsubsection{zusätzliche Berechnungen bei einem Dreierteam}
\begin{tabular}[h]{ll}
H3 vs. G1 & 2 Berechnungen \\
H3 vs. G2 & 2 Berechnungen \\
\end{tabular}
\\
\\Vier zusätzliche Berechnungen treten bei einem Dreierteam auf. Das Beispiel trifft analog für ein Dreierteam der Gastmannschaft zu.
\subsubsection{zusätzliche Berechnung bei zwei Dreierteams}
\begin{tabular}[h]{ll}
H3 vs. G3 & 2 Berechnungen \\
\end{tabular}
\\Zwei zusätzliche Berechnungen treten bei zwei Dreierteams auf.
\subsubsection{Maximalanzahl der Berechnungen}
Somit sind für ein Spiel im Beerpongturnier maximal 18 Elo-Berechnungen pro Spiel nötig. Dies ergibt bei durchschnittliche 25 Spielen 450 Elo-Berechnungen.

\subsection{Kategorien}
\begin{tabular}{|l|l|l|}
\hline
\thead{\textbf{Elo-Zahl}} & \thead{\textbf{Titel}} & \thead{\textbf{Erklärung} \\ \textbf{und Voraussetzungen}} \\
\hline
\makecell{>2400} & \makecell{ \textbf{Super-Großmeister (SGM)}} & \makecell{mind. 5x BPT Champion} \\
\hline
\makecell{2100 bis 2399} & \makecell{\textbf{Großmeister (GM)}} & \makecell{mind. 2x BPT Champion} \\
\hline
\makecell{2000 bis 2099} & \makecell{\textbf{Beerpong-Meister (BM)}} & \makecell{mind. 6x Top 3}\\
\hline
\makecell{1900 bis 1999} & \makecell{\textbf{Pro}} & \makecell{}\\
\hline
\makecell{1801 bis 1899} & \makecell{\textbf{Experte}} & \makecell{}\\
\hline
\makecell{1700 bis 1800} & \makecell{\textbf{Amateuer Klasse A}} & \makecell{Verlust des \textbf{SGM-Titel}\\ bei aktueller Elo \\in dieser Stufe}\\
\hline
\makecell{1600 bis 1699} & \makecell{\textbf{Amateur Klasse B}} & \makecell{Verlust des \textbf{GM-Titel}\\ bei aktueller Elo \\in dieser Stufe} \\
\hline
\makecell{1400 bis 1599} & \makecell{\textbf{Amateur Klasse C}} & \makecell{Verlust des \textbf{BM-Titel}\\ bei aktueller Elo \\in dieser Stufe}\\
\hline
\makecell{1000 bis 1399} & \makecell{\textbf{Gelegenheitsspieler}} & \makecell{Verlust des \textbf{Pro-Titel}\\ bei aktueller Elo \\in dieser Stufe}\\
\hline
\makecell{<1000} & \makecell{\textbf{Anfänger}} & \makecell{Verlust des \textbf{Experten-Titel}\\ bei aktueller Elo \\in dieser Stufe}\\
\hline
\end{tabular}
\end{document}
